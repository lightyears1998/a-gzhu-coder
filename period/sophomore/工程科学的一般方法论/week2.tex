\documentclass[UTF8]{ctexart}

\title{%
  如何解决大学城学生闯红灯的问题?\\
  \large 工程科学的一般方法论小组讨论 Week 2}

\author{
谢金宏 \and 胡涛 \and 何汉根 \and 房华恒
\and 吴厚锋 \and 李俊民 \and 陈淇铭 \and 陈树康
}

\begin{document}

\maketitle

在“大学城学生闯红灯”这个情景中,“学生行人”、“过路车辆”和“交通信号灯”三个要素组成一个系统。我们将系统边界限定于某一特定十字路口附近,这样,系统输入是即将进入该十字路口的学生和车辆;定义该系统的系统响应是学生是否闯红灯。显然,该系统属于时变系统。

从控制论的角度分析,此系统采取开环控制,系统中没有传感器和执行器,交通信号灯是控制器,产生的控制信号目的是将系统响应控制为阴性的正常状态。

实际生活中系统响应时常为阳性失控状态。系统失控源于:a. 学生安全意识薄弱,或因为赶时间上课/工作;b. 交通信号的固定时长不合理,行人难以在一个绿灯时长内通过路口,部分路口(如菊苑)道路两旁的红绿灯不同步;c. 过路车辆较少属于诱因(根源仍是a. 学生安全意识薄弱)。

基于以上分析,我们提出以下解决方案:a. 教育学生提高安全意识,惩戒闯红灯行为。b. 基于系统的开环控制和时变特性,统计不同时段的行人流量规律取得先验经验,在不同时段根据人流设置合理的绿灯时长。

\end{document}
