\documentclass[UTF8]{ctexart}
\usepackage{graphicx}
\usepackage{hyperref}

\title{%
  现代基建所产生的废弃物的处理\\
  \large 第3小组}

\author{
谢金宏 \and 胡涛 \and 何汉根 \and 房华恒
\and 奚厚铧 \and 李俊民 \and 陈淇铭 \and 陈树康
}

\begin{document}

\maketitle

\section{现代基建的类型}


基础设施是指为社会生产和居民生活提供公共服务的物质工程设施,是用于保证国家或地区社会经济活动正常进行的公共服务系统。它是社会赖以生存发展的一般物质条件。现代的基础设施包括交通、邮电、能源动力、供水供电、商业服务、科研与技术服务、园林绿化、环境保护、文化教育、卫生事业等市政公用工程设施和居住建筑、办公商用建筑等公共生活服务设施。例如,基础道路、5G通信基站、核能发电站都属于基础设施的范畴。

\section{现代基建中产生的废弃物}

常见的建筑废物有建筑物建设、铺设、修缮或拆除过程中产生的渣土、弃土、弃料和淤泥等等。按产生源进行分类,可将建筑废物可分为工程渣土、装修垃圾、拆迁垃圾、工程泥浆等;按组成成分分类,建筑垃圾中可分为渣土、混凝土块、碎石块、砖瓦碎块、废砂浆、泥浆、沥青块、废塑料、废金属、废竹木等。这些废弃物污染建筑周边的环境,对建筑本身是没有帮助的,只有妥善地对这些废弃物进行处理,才能实现理想的工程项目建设的目标。

\section{废弃物对应的处理方案}

建筑废物可分为可回收废弃物和不可回收废弃物。建筑废物经分拣、剔除或粉碎后,大多可以作为再生资源重新利用。

建筑废物按照处理的难易程度,又将可回收利用废弃物分为易回收废弃物和难回收废弃物两种。对建筑废弃物的处理,主要有以下几种处理方案:

\begin{enumerate}
\item 利用废弃建筑混凝土和废弃砖石生产粗细骨料,可用于生产相应强度等级的混凝土、砂浆或制备诸如砌块、墙板、地砖等建材制品。粗细骨料添加固化类材料后,也可用于公路路面基层。
\item 利用废砖瓦生产骨料,可用于生产再生砖、砌块、墙板、地砖等建材制品。
\item 渣土可用于筑路施工、桩基填料、地基基础等。
\item 对于废弃木材类建筑垃圾,尚未明显破坏的木材可以直接再用于重建建筑,破损严重的木质构件可作为木质再生板材的原材料或造纸等。
\item 废弃路面沥青混合料可按适当比例直接用于再生沥青混凝土。
\item 废弃道路混凝土可加工成再生骨料用于配制再生混凝土。
\item 废钢材、废钢筋及其他废金属材料可直接再利用或回炉加工。
\item 废玻璃、废塑料、废陶瓷等建筑垃圾视情况区别利用。
\end{enumerate}

\section{参考资料}

\begin{enumerate}
\item 百度百科上的“基础设施”词条  \url{https://baike.baidu.com/item/%E5%9F%BA%E7%A1%80%E8%AE%BE%E6%96%BD/3831695}
\item 百度百科上的“建筑垃圾”词条  \url{https://baike.baidu.com/item/%E5%BB%BA%E7%AD%91%E5%9E%83%E5%9C%BE/4565765}
\end{enumerate}

\end{document}
