\documentclass[UTF8]{ctexart}
\usepackage{graphicx}

\title{元胞自动机相关论文读后感}

\author{软件171 谢金宏}

\begin{document}

\maketitle

\section{《复杂网络上病毒传播的元胞自动机模拟》读后感}

\paragraph{元胞邻居的确定} 这篇论文对我的启发是一个元胞的邻居不仅可以由空间位置相邻来决定,还可以人为地定义元胞的邻居。论文中元胞空间定义在数据结构图中,若存在从结点i指向结点j的有向边,则可以说结点j的一个邻居是结点i。

\paragraph{一个可能的演化规则的改进} 论文中使用了SIR和SIS演化规则。这两种演化规则中都不包含免疫病毒感染的结点向网络中广播免疫方法的规则。而在现实世界中,可能出现某一个结点研究出了免疫病毒的方案而将免疫方案在整个网络上传播的现象。如果新增定义结点状态P为获悉免疫方案的结点,而其他非P结点若其邻居中有P结点,则存在一定概率从该结点处获得免疫方法的演化规则。或许可以得到有趣的结果。

\section{《基于元胞自动机模型的土地利用变化模拟》读后感}

\paragraph{通过实测数据来生成元胞自动机的演化规则} 元胞自动机的演化规则可以对实际测量得到的数据中生成,这也算是大数据的应用。元胞自动机的优势在于能用简单的演化规则来模拟复杂系统的变化。但是演化规则的确定比较难。正如论文中所提到的,一些研究使用人工智能方法通过对实测数据的分析来获得演化规则,但这样获得的演化规则不容易发现隐藏在“空间格局下的变化规律”。而不使用人工智能的分析方法中,对大量数据的分析能力又较弱。而此论文使用人工方法较成功地研究了多个维度的目标,可以成为这一方面的参考范例。

\section{《江河源区高寒草甸退化序列上“秃斑”连通效应的元胞自动机模拟》读后感}

这篇论文中的模拟与现实情况的吻合程度是五篇论文中最好的,这应该得益于完整的研究流程:数据采集、建立模型迭代调整、分析结果。但这篇论文的数据收集环节不像《基于元胞自动机模型的土地利用变化模拟》采集了多个年度长时间段的数据,而只是采集了一个特定时段的信息,模拟的结果甚至比《土地变化模拟》更贴合实际。可见信息的采集不是越多越好,而是需要根据研究对象的特性,有针对性进行采集。

\section{《空间复杂性与地理元胞自动机模拟研究》读后感}

这篇论文是五篇论文中时间最早(1999年)的论文,或许是元胞自动机方法在中国推广的先行者。相比起其他四篇时间更晚的论文。论文把重点放在了介绍了元胞自动机的概念和特征,以及用于研究复杂地理系统的可行性和必然性、地理元胞自动机对标准元胞自动机的拓展等。特别是指出了地理元胞自动机对标准元胞自动机的拓展,并提出了地理信息系统与地理元胞自动系统的耦合方式等开创先河之举。

\section{《岩石破坏演化细观非均质物理元胞自动机模拟研究》读后感}

这篇论文体现了使用元胞自动机来研究非线性的系统相互作用时的强大之处。即使采用简单的摩尔八邻居模型,只要提出较为符合现实情况的元胞演化规则,就能得到与现实情况符合得较好的研究结果。另外本篇论文别出心裁地从能量的角度出发进行研究,避开了较为复杂的牛顿力学方式,也得到了很好的结果。可见在进行研究时,如果利用传统的经典的方式进行研究后遇到了困难,可以改换研究思路和方法,或许有意想不到的收获。

\end{document}
